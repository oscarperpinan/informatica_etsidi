% Created 2022-02-02 mié 21:03
% Intended LaTeX compiler: pdflatex
\documentclass[xcolor={usenames,svgnames,dvipsnames}, aspectratio=169]{beamer}
\usepackage[utf8]{inputenc}
\usepackage[T1]{fontenc}
\usepackage{graphicx}
\usepackage{grffile}
\usepackage{longtable}
\usepackage{wrapfig}
\usepackage{rotating}
\usepackage[normalem]{ulem}
\usepackage{amsmath}
\usepackage{textcomp}
\usepackage{amssymb}
\usepackage{capt-of}
\usepackage{hyperref}
\usepackage{color}
\usepackage{listings}
\usepackage[spanish]{babel}
\usecolortheme{rose}
\setbeamercolor{alerted text}{fg=DarkBlue}
\setbeamerfont{alerted text}{series=\bfseries}
\setbeamerfont{block title}{series=\bfseries}
\setbeamercolor{block title}{bg=structure.fg!20!bg!50!bg}
\setbeamercolor{block body}{use=block title,bg=block title.bg}
\setbeamertemplate{navigation symbols}{\insertsectionnavigationsymbol}
\AtBeginSection[]{\begin{frame}[plain]\tableofcontents[currentsection,sectionstyle=show/shaded]\end{frame}}
\AtBeginSubsection[]{\begin{frame}[plain]\tableofcontents[currentsubsection,sectionstyle=show/shaded,subsectionstyle=show/shaded]\end{frame}}
\lstset{keywordstyle=\color{blue}, commentstyle=\color{gray!90}, basicstyle=\ttfamily\small, columns=fullflexible, breaklines=true,linewidth=\textwidth, backgroundcolor=\color{gray!23}, basewidth={0.5em,0.4em}, literate={¡}{{\textexclamdown}}1 {á}{{\'a}}1 {ñ}{{\~n}}1 {é}{{\'e}}1 {ó}{{\'o}}1 {í}{{\'i}}1 {ú}{{\'u}}1 {º}{{\textordmasculine}}1, showstringspaces=false}
\usepackage{mathpazo}
\usepackage{siunitx}
\hypersetup{colorlinks=true, linkcolor=Blue, urlcolor=Blue}
\usepackage{fancyvrb}
\DefineVerbatimEnvironment{verbatim}{Verbatim}{fontsize=\tiny, formatcom = {\color{black!70}}}
\setbeamertemplate{footline}[frame number]
\usetheme{Boadilla}
\usefonttheme{serif}
\author{Oscar Perpiñán Lamigueiro}
\date{}
\title{Tema 5: Punteros}
\hypersetup{
 pdfauthor={Oscar Perpiñán Lamigueiro},
 pdftitle={Tema 5: Punteros},
 pdfkeywords={},
 pdfsubject={},
 pdfcreator={Emacs 27.1 (Org mode 9.4.6)}, 
 pdflang={Spanish}}
\begin{document}

\maketitle

\section{Definición}
\label{sec:orga485964}

\begin{frame}[label={sec:org35a2e3c}]{Datos y Memoria}
\begin{itemize}
\item Los datos de un programa se almacenan en la memoria del ordenador.

\item La memoria del ordenador está estructurada en \alert{bytes} (8 bits).

\item Cada byte tiene una posición en la memoria (dirección).
\end{itemize}

\begin{center}
  \begin{tabular}{ | c | c}
    \cline{1-1}
     & Posición de memoria N \\ \cline{1-1}
     & \\ \cline{1-1}
     & \\ \cline{1-1}
    $\leftarrow$  1 byte $\rightarrow$ & \vdots \\ \cline{1-1}
     & \\ \cline{1-1}
     & Posición de memoria 1 \\ \cline{1-1}
     & Posición de memoria 0 \\ \cline{1-1}
  \end{tabular}
\end{center}
\end{frame}

\begin{frame}[label={sec:orgc3678ec},fragile]{Dirección de memoria de una variable}
 \begin{itemize}
\item Ejemplo: un dato \texttt{int} ocupa 4 bytes.
\end{itemize}
\lstset{language=C,label= ,caption= ,captionpos=b,numbers=none}
\begin{lstlisting}
int x;
\end{lstlisting}

\begin{center}
  \begin{tabular}{ | c | c}
    \cline{1-1}
     & Dirección\\ \cline{1-1}
     & \\ \hline
     $\uparrow$ & 1245051\\ \cline{1-1}
     & 1245050 \\ \cline{1-1}
     x     & 1245049\\ \cline{1-1}
     $\downarrow$     & \textbf{1245048} \\ \hline
     & \\ \cline{1-1}
  \end{tabular}
\end{center}
\end{frame}

\begin{frame}[label={sec:org83a456c},fragile]{Operador \&}
 El operador \texttt{\&} (\emph{ampersand}) aplicado a una variable cualquiera proporciona su dirección de memoria.

\lstset{language=C,label= ,caption= ,captionpos=b,numbers=none}
\begin{lstlisting}
#include <stdio.h>

int main()
{
  // No hace falta asignar valor inicial
  // para que la variable
  // tenga dirección de memoria
  int x; 

  printf("La variable x está almacenada en %lli.\n",
	 &x);

  return 0;
}
\end{lstlisting}
\end{frame}

\begin{frame}[label={sec:org7b7f629}]{¿Qué es un puntero?}
\begin{block}{Un puntero apunta a una variable}
Un \alert{puntero} (\emph{pointer}) es una \alert{variable} (tipo número entero) que contiene la dirección de memoria de una variable: 

\begin{itemize}
\item El puntero es una referencia de la variable a la que apunta.

\item El valor del puntero es la dirección de memoria de la variable.

\item La variable está apuntada por el puntero.
\end{itemize}
\end{block}
\end{frame}

\section{Uso de punteros}
\label{sec:orgd74cc24}
\begin{frame}[label={sec:org4e464a5},fragile]{Declaración de un puntero}
 Un puntero se declara: 
\begin{itemize}
\item Indicando el tipo de datos de la variable a la que apunta.
\item Incluyendo un asterisco \texttt{*} antes del identificador.
\end{itemize}

\lstset{language=C,label= ,caption= ,captionpos=b,numbers=none}
\begin{lstlisting}
int main()
{
  // p1: puntero apuntando a una variable de tipo entero
  int *p1;
  // p2: puntero apuntando a una variable de tipo caracter
  char *p2;
  // p3: puntero apuntando a una variable de tipo real
  float *p3;
  // p4: puntero apuntando a una variable genérica
  void *p4;

  return 0;
}

\end{lstlisting}
\end{frame}

\begin{frame}[label={sec:orgde55108},fragile]{Un puntero es una variable \texttt{int}}
 \begin{itemize}
\item El contenido de una variable puntero es la dirección de memoria, un valor de tipo entero.
\item Su tamaño depende del sistema:
\begin{itemize}
\item Sistemas de 32 bits ocupan 4 bytes.
\item Sistemas de 64 bits ocupan 8 bytes.
\end{itemize}
\end{itemize}

\lstset{language=C,label= ,caption= ,captionpos=b,numbers=none}
\begin{lstlisting}
#include <stdio.h>

int main()
{
  int x, *p;
  // sizeof devuelve el numero de bytes de una variable o tipo de datos
  printf("La variable x ocupa %i bytes.\n", 
	 sizeof(x));
  printf("El puntero p ocupa %i bytes.\n",
	 sizeof(p));

  return 0;
}

\end{lstlisting}
\end{frame}

\begin{frame}[label={sec:org68f25af},fragile]{Asignación}
 \lstset{language=C,label= ,caption= ,captionpos=b,numbers=none}
\begin{lstlisting}
int main()
{
  int x, y;
  // p1 apunta a x
  int *p1 = &x, *p2, *p3;
  // p2 apunta a y
  p2 = &y;
  // p3 apunta a la misma variable que p1, es decir, x
  p3 = p1;

  return 0;
}
\end{lstlisting}
\end{frame}

\begin{frame}[label={sec:org2907eca},fragile]{Operador \texttt{*}}
 El operador \texttt{*} aplicado a un puntero proporciona el valor de la variable apuntada por el puntero.

\lstset{language=C,label= ,caption= ,captionpos=b,numbers=none}
\begin{lstlisting}
#include <stdio.h>

int main()
{
  int x = 10, *p;
  // Operador & para obtener la direccion de x
  p = &x;
  // *p y x son lo mismo
  printf("La variable apuntada vale %i",
	 *p);
}
\end{lstlisting}
\end{frame}

\begin{frame}[label={sec:org001412c}]{Operaciones con punteros}
\begin{itemize}
\item La \alert{suma o resta de un entero a un puntero} produce una \alert{nueva localización de memoria}.
\item Se pueden \alert{comparar punteros} utilizando expresiones lógicas para \alert{comprobar si  apuntan a la misma dirección de memoria}.
\item La \alert{resta de dos punteros} da como resultado el \alert{número de variables entre las dos direcciones}.
\end{itemize}
\end{frame}

\section{Punteros a vectores}
\label{sec:org50a7a3b}

\begin{frame}[label={sec:org212487f},fragile]{Punteros y vectores}
 El identificador de un vector es un puntero \emph{constante} que apunta al primer elemento del vector.

\lstset{language=C,label= ,caption= ,captionpos=b,numbers=none}
\begin{lstlisting}
#include <stdio.h>

int main()
{
  int vector[3] = {1, 2, 3};
  int *p;
  // p apunta al primer elemento del vector
  p = &vector[0];
  printf("El primer elemento es %i\n", *p);
  // De forma mas concisa
  p = vector;
  printf("El primer elemento es %i\n", *p);

  return 0;
}
\end{lstlisting}
\end{frame}

\begin{frame}[label={sec:org8d7c7be},fragile]{Recorrido de un vector}
 Podemos recorrer un vector a través de su puntero con sumas y restas

\lstset{language=C,label= ,caption= ,captionpos=b,numbers=none}
\begin{lstlisting}
#include <stdio.h>
int main()
{
  int i, vector[3] = {1, 2, 3};
  int *p, *p1;
  p = vector;
  // p apunta a vector[0].
  printf("%i\t", *p);
  // p + 1 apunta a vector[1]
  p1 = p + 1;
  printf("%i\t", *p1);
  // *(p + 1) es equivalente a v[i + 1]
  printf("%i\t", *(p + 1));
  printf("%i\t", vector[1]);
  return 0;
}
\end{lstlisting}
\end{frame}



\begin{frame}[label={sec:org42bf517},fragile]{Modificación de un vector}
 Podemos modificar un vector a través de su puntero

\lstset{language=C,label= ,caption= ,captionpos=b,numbers=none}
\begin{lstlisting}
#include <stdio.h> 
int main(){
  int i, vector[3];
  int *p;
  p = vector;
  //vector[0] = 1
  *p = 1;
  //vector[1] = 2
  *(p + 1) = 2;
  //vector[2] = 3
  *(p + 2) = 3;

  printf("El vector es %i, %i, %i.\n",
	 vector[0], vector[1], vector[2]);
  
  return 0;
}
\end{lstlisting}
\end{frame}


\begin{frame}[label={sec:org2cf2319},fragile,plain]{Aritmética de punteros con vectores}
 \lstset{language=C,label= ,caption= ,captionpos=b,numbers=none}
\begin{lstlisting}
#include <stdio.h>
#define N 10

int main(){
  int vector[N] = {1};
  int *pVec, *pFin;
  // Puntero apuntando al segundo elemento
  pVec = vector + 1;
  // Puntero apuntando al ultimo elemento
  pFin = vector + N - 1;
  // Comparamos los punteros para avanzar
  while (pVec <=  pFin)
    { // vector[i] = vector[i - 1] + 1
      *pVec = *(pVec - 1) + 1; 
      printf("%i\t", *pVec);
      ++pVec; // Movemos el puntero por el vector
    }
  return 0;
} 
\end{lstlisting}
\end{frame}

\section{Punteros a cadenas de caracteres}
\label{sec:org834306e}

\begin{frame}[label={sec:org1f0468d},fragile]{Punteros y cadenas}
 El identificador de una cadena es un puntero \emph{constante} que apunta al primer elemento.

\lstset{language=C,label= ,caption= ,captionpos=b,numbers=none}
\begin{lstlisting}
#include <stdio.h>

int main()
{
  char letras[3] = {'a', 'b', 'c'};
  char *p;
  // p apunta al primer elemento 
  p = &letras[0];
  printf("El primer elemento es %c\n", *p);
  // De forma mas concisa
  p = letras;
  printf("El primer elemento es %c\n", *p);

  return 0;
}
\end{lstlisting}
\end{frame}

\begin{frame}[label={sec:orgf2063e0},fragile]{Recorrido de una cadena}
 \lstset{language=C,label= ,caption= ,captionpos=b,numbers=none}
\begin{lstlisting}
#include <stdio.h>
int main()
{
  char mensaje[] = "Hola Mundo";
  char *p = mensaje;
  int i = 0;
  // Movemos el puntero por la cadena
  while(*p != '\0')
    {
      printf("%c", *p);
      p++; // Incrementa el puntero para 
    }      // pasar al siguiente caracter
  printf("\n");
}
\end{lstlisting}
\end{frame}

\begin{frame}[label={sec:org6e29f03},fragile]{Aritmética de punteros con cadenas}
 \lstset{language=C,label= ,caption= ,captionpos=b,numbers=none}
\begin{lstlisting}
#include <stdio.h>

int main(){
  char texto[] = "Hola Mundo";
  char *pChar = texto;

  while (*pChar != '\0')
    ++pChar; // Movemos el puntero por la cadena

  // El identificador "texto" es un puntero que
  // apunta al primer caracter de la cadena.
  // Si lo restamos del puntero movil
  // tenemos el total.
  printf("La cadena tiene %i caracteres.\n",
	 pChar - texto);

  return 0;
}
\end{lstlisting}
\end{frame}

\section{Punteros a estructuras}
\label{sec:orgdfd423d}

\begin{frame}[label={sec:orgca53829},fragile,plain]{Punteros a estructuras}
 \begin{itemize}
\item Un puntero a una estructura se declara igual que un puntero a un tipo simple.
\item Para acceder a un miembro de la estructura se emplea el operador \texttt{->}.
\end{itemize}
\lstset{language=C,label= ,caption= ,captionpos=b,numbers=none}
\begin{lstlisting}
#include <stdio.h>
typedef struct 
{
  int y, m, d;
} fecha;

int main(){
  fecha f = {2000, 10, 15}, *p;
  // p apunta a la estructura
  p = &f;

  printf("%i-%i-%i",
	 p->d, p->m, p->y);

  return 0;
}
\end{lstlisting}
\end{frame}

\begin{frame}[label={sec:org6ce3372},fragile,plain]{Ejemplo}
 \lstset{language=C,label= ,caption= ,captionpos=b,numbers=none}
\begin{lstlisting}
#include <stdio.h>
typedef struct 
{
  int y, m, d;
} fecha;

int main()
{
  fecha f, *p = &f;
  // Rellenamos la estructura a través de su puntero
  p->d = 15;
  p->m = 10;
  p->y = 2000;

  printf("%i-%i-%i",
	 f.d, f.m, f.y);

  return 0;
}
\end{lstlisting}
\end{frame}

\begin{frame}[label={sec:orgca950ae},fragile,plain]{Ejemplo con scanf}
 \lstset{language=C,label= ,caption= ,captionpos=b,numbers=none}
\begin{lstlisting}
#include <stdio.h>
typedef struct 
{
  int y, m, d;
} fecha;

int main()
{
  fecha f, *p = &f;
  // Rellenamos la estructura a través de su puntero con
  // scanf. Atención al uso de &.
  scanf("%i", &(p->d));
  scanf("%i", &(p->m));
  scanf("%i", &(p->y));

  printf("%i-%i-%i",
	 f.d, f.m, f.y);

  return 0;
}
\end{lstlisting}
\end{frame}


\section{Funciones y punteros}
\label{sec:orgc017a4d}
\begin{frame}[label={sec:orgb07a3f6}]{Paso por referencia}
\begin{itemize}
\item El uso de punteros en funciones permite el \alert{paso por referencia}. De esta forma la función puede acceder (y \alert{modificar}) a la variable original (\emph{sin copia}).

\item Las funciones que emplean \alert{vectores y cadenas} como argumentos funcionan con \alert{paso por referencia} (\emph{el identificador de un vector es un puntero}).
\end{itemize}
\end{frame}

\begin{frame}[label={sec:org0da62e0},fragile,plain]{Ejemplo}
 \lstset{language=C,label= ,caption= ,captionpos=b,numbers=none}
\begin{lstlisting}
#include <stdio.h>
void operaciones (float x, float y,
		  float *s, float *p, float *d);
int main(){
  float a = 1.0, b = 2.0; // Datos
  float suma, producto, division; // Resultados
  operaciones(a, b, &suma, &producto, &division);
  printf("S: %f \t P: %f \t D: %f \t",
	 suma, producto, division);
  return 0;
}
//Funcion con varios resultados
void operaciones (float x, float y,
		  float *s, float *p, float *d)
{// Cada puntero sirve para un resultado
  *s = x + y;
  *p = x * y;
  *d = x / y;
}
\end{lstlisting}
\end{frame}

\section{Asignación dinámica de memoria}
\label{sec:org1f0590b}

\begin{frame}[label={sec:org0069d21},fragile]{\texttt{malloc} y \texttt{free}}
 \begin{itemize}
\item La \alert{asignación dinámica} de memoria permite definir \alert{objetos (p.ej. vectores) de dimensión variable}.
\item La función \texttt{malloc} permite asignar, durante la ejecución del programa, un bloque de memoria de n bytes consecutivos para almacenar los datos (devuelve \texttt{NULL} si no es posible la asignación)
\item La función \texttt{free} permite liberar un bloque de memoria previamente asignado.
\end{itemize}
\end{frame}

\begin{frame}[label={sec:org50f278d},fragile]{Uso de \texttt{malloc} y \texttt{free}}
 \lstset{language=C,label= ,caption= ,captionpos=b,numbers=none}
\begin{lstlisting}
int *pInt;
...
// Reservamos la memoria suficiente para almacenar 
// un int y asignamos su dirección a pInt
pInt = malloc(sizeof(int));

// Comprobamos si la asignación 
// se ha realizado correctamente
if (pInt  == NULL) {
      printf("Error: memoria no disponible.\n");
      exit(-1);
}

... // Codigo usando el puntero

free(pInt); // Liberamos memoria al terminar
\end{lstlisting}
\end{frame}

\begin{frame}[label={sec:org5ed60f1},fragile,plain]{Ejemplo}
 \lstset{language=C,label= ,caption= ,captionpos=b,numbers=none}
\begin{lstlisting}
#include <stdio.h>
#include <stdlib.h> //Necesaria para malloc y free
int main () {
  int *vec, i, N = 100;
  vec = malloc(sizeof(int) * N);
  //Comprueba si malloc ha funcionado
  if (vec == NULL) {
      printf("Error: memoria no disponible.\n");
      exit(-1);
    }
  // El resultado es un puntero-vector de N elementos
  for (i = 0; i < N; ++i) 
    vec[i] = i * i; // Rellenamos el puntero-vector
  for (i = 0; i < N; ++i) // Mostramos contenido
    printf("%i \t", *(vec + i));
  free(vec); // Liberamos el puntero-vector
  return 0;
}
\end{lstlisting}
\end{frame}
\end{document}