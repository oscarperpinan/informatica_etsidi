% Created 2022-02-02 mié 21:03
% Intended LaTeX compiler: pdflatex
\documentclass[xcolor={usenames,svgnames,dvipsnames}, aspectratio=169]{beamer}
\usepackage[utf8]{inputenc}
\usepackage[T1]{fontenc}
\usepackage{graphicx}
\usepackage{grffile}
\usepackage{longtable}
\usepackage{wrapfig}
\usepackage{rotating}
\usepackage[normalem]{ulem}
\usepackage{amsmath}
\usepackage{textcomp}
\usepackage{amssymb}
\usepackage{capt-of}
\usepackage{hyperref}
\usepackage{color}
\usepackage{listings}
\usepackage[spanish]{babel}
\usecolortheme{rose}
\setbeamercolor{alerted text}{fg=DarkBlue}
\setbeamerfont{alerted text}{series=\bfseries}
\setbeamerfont{block title}{series=\bfseries}
\setbeamercolor{block title}{bg=structure.fg!20!bg!50!bg}
\setbeamercolor{block body}{use=block title,bg=block title.bg}
\setbeamertemplate{navigation symbols}{\insertsectionnavigationsymbol}
\AtBeginSection[]{\begin{frame}[plain]\tableofcontents[currentsection,sectionstyle=show/shaded]\end{frame}}
\AtBeginSubsection[]{\begin{frame}[plain]\tableofcontents[currentsubsection,sectionstyle=show/shaded,subsectionstyle=show/shaded]\end{frame}}
\lstset{keywordstyle=\color{blue}, commentstyle=\color{gray!90}, basicstyle=\ttfamily\small, columns=fullflexible, breaklines=true,linewidth=\textwidth, backgroundcolor=\color{gray!23}, basewidth={0.5em,0.4em}, literate={¡}{{\textexclamdown}}1 {á}{{\'a}}1 {ñ}{{\~n}}1 {é}{{\'e}}1 {ó}{{\'o}}1 {í}{{\'i}}1 {ú}{{\'u}}1 {º}{{\textordmasculine}}1, showstringspaces=false}
\usepackage{mathpazo}
\usepackage{siunitx}
\hypersetup{colorlinks=true, linkcolor=Blue, urlcolor=Blue}
\usepackage{fancyvrb}
\DefineVerbatimEnvironment{verbatim}{Verbatim}{fontsize=\tiny, formatcom = {\color{black!70}}}
\setbeamertemplate{footline}[frame number]
\usetheme{Boadilla}
\usefonttheme{serif}
\author{Oscar Perpiñán Lamigueiro}
\date{}
\title{Tema 6: Funciones}
\hypersetup{
 pdfauthor={Oscar Perpiñán Lamigueiro},
 pdftitle={Tema 6: Funciones},
 pdfkeywords={},
 pdfsubject={},
 pdfcreator={Emacs 27.1 (Org mode 9.4.6)}, 
 pdflang={Spanish}}
\begin{document}

\maketitle

\section{Definición de función}
\label{sec:org11397a3}
\begin{frame}[label={sec:org359f385}]{¿Qué es una función?}
\begin{block}{}
Una función es un bloque de código que realiza una tarea determinada a partir de unos datos.
\end{block}
\begin{block}{Ventajas}
\begin{itemize}
\item Permiten \alert{programación estructurada y abstracta}, sin necesidad de conocer
el detalle de la implementación de una tarea concreta.
\item Mejoran la \alert{legibilidad} del código.
\item Facilitan el \alert{mantenimiento} del programa.
\item Permiten \alert{reutilizar código} de manera eficiente (\alert{DRY!}).
\end{itemize}
\end{block}
\end{frame}

\begin{frame}[label={sec:org92612dd},fragile]{¿Cómo se declara una función?}
 \begin{block}{Prototipo de una función:}
\begin{enumerate}
\item Tipo de valor que devuelve (\texttt{int}, \texttt{void}, \ldots{})
\item Nombre de la función (debe ser un identificador válido y \alert{útil}).
\item Lista de argumentos que emplea, por tipo y nombre (puede estar vacía).
\end{enumerate}

\lstset{language=C,label= ,caption= ,captionpos=b,numbers=none}
\begin{lstlisting}
tipo nombre_funcion(tipo1 arg1, tipo2 arg2, ...);
\end{lstlisting}
\end{block}
\begin{block}{Ejemplos}
\lstset{language=C,label= ,caption= ,captionpos=b,numbers=none}
\begin{lstlisting}
void printHello(int veces);

float areaTriangulo(float b, float h);
\end{lstlisting}
\end{block}
\end{frame}
\begin{frame}[label={sec:orge4a3c88},fragile]{¿Cómo se define una función?}
 \lstset{language=C,label= ,caption= ,captionpos=b,numbers=none}
\begin{lstlisting}
// Definicion de la funcion printHello

// No devuelve nada (void)

// Necesita un argumento llamado veces,
// un entero (int), para funcionar.

void printHello(int veces)
{
  int i;
  for (i = 1; i <= veces; i++)
    printf("Hello World!\n");
}
\end{lstlisting}
\end{frame}
\begin{frame}[label={sec:orged7b549},fragile]{¿Cómo se define una función?}
 \lstset{language=C,label= ,caption= ,captionpos=b,numbers=none}
\begin{lstlisting}
// Definicion de la funcion areaTriangulo

// Devuelve un real (float)

// Necesita dos argumentos, b y h, reales.

float areaTriangulo(float b, float h)
{
  float area;

  area = b * h / 2.0;

  return area;
}
\end{lstlisting}
\end{frame}

\section{Estructura de un programa}
\label{sec:orgfd8b3e2}

\begin{frame}[label={sec:org49ae446},fragile]{Estructura de un programa}
 \begin{itemize}
\item \emph{Puede} incluir directivas de inclusión (\texttt{include}).
\item \emph{Puede} incluir directivas de sustitución (\texttt{define}).
\item Declaración de funciones (prototipo).
\item Todos los programas tienen al menos una función: \texttt{main}.
\item Definición de las funciones.
\end{itemize}
\end{frame}


\begin{frame}[label={sec:orgce2e879},fragile]{Directivas de inclusión \texttt{include}}
 Permiten incluir cabeceras (definiciones) procedentes de otros archivos

\lstset{language=C,label= ,caption= ,captionpos=b,numbers=none}
\begin{lstlisting}
//Librerias del sistema
#include <stdio.h>
#include <math.h>
//Librerias propias del desarrollador
#include "myHeader.h"
\end{lstlisting}
\end{frame}


\begin{frame}[label={sec:org542126c},fragile]{Directivas de sustitución \texttt{define}}
 \begin{itemize}
\item \texttt{define} permite definir símbolos que serán sustituidos por su valor.
\end{itemize}
\lstset{language=C,label= ,caption= ,captionpos=b,numbers=none}
\begin{lstlisting}
#include <stdio.h>
//Habitualmente con mayúsculas
//Atención: SIN signo igual NI punto y coma
#define PI 3.141592

int main()
{
  float r = 2.0;
  printf("Una circunferencia de radio %f", r);
  printf(" tiene un area de %f", PI * r * r);
  return 0;
}
\end{lstlisting}

\begin{itemize}
\item \texttt{undef} elimina la definición del símbolo.
\end{itemize}
\lstset{language=C,label= ,caption= ,captionpos=b,numbers=none}
\begin{lstlisting}
#undef PI
\end{lstlisting}
\end{frame}


\begin{frame}[label={sec:org5167c5b},fragile,plain]{Declaración y definición de funciones}
 \lstset{language=C,label= ,caption= ,captionpos=b,numbers=none}
\begin{lstlisting}
#include <stdio.h>
// Prototipo de la función (termina en ;)
void printHello(int n);

// Función main
int main() {
  //Uso de la función en main
  printHello(3);
  return 0;
}
// Definición de la función
void printHello(int n)
{
  int i;
  for (i = 1; i <= n; i++)
    printf("Hello World!\n");
}
\end{lstlisting}
\end{frame}


\begin{frame}[label={sec:orgf7eaa07},fragile,plain]{Declaración y definición de funciones}
 \lstset{language=C,label= ,caption= ,captionpos=b,numbers=none}
\begin{lstlisting}
#include <stdio.h>
// Prototipo de la función (termina en ;)
float areaTriangulo(float b, float h);
// Función main
int main(){
  float at;
  //Uso de la función en main
  at = areaTriangulo(1, 2);
  printf("%f", at);
  return 0;
}
// Definición de la función
float areaTriangulo(float b, float h)
{
  float area;
  area = b * h / 2.0;
  return area;
}
\end{lstlisting}
\end{frame}

\section{Ámbito de una variable}
\label{sec:org149900e}

\begin{frame}[label={sec:orgd4529b5},fragile,plain]{Variables globales}
 \lstset{language=C,label= ,caption= ,captionpos=b,numbers=none}
\begin{lstlisting}
#include <stdio.h>

int gVar = 3; //Variable global

void foo(void);

int main(){
  printf("main (1):\t gVar es %d.\n", gVar);
  foo();
  gVar *= 2;
  printf("main (2):\t gVar es %d.\n", gVar);
  return 0;
}

void foo(void){
  gVar = gVar + 1;
  printf("foo:\t gVar es %d.\n", gVar);
}
\end{lstlisting}
\end{frame}

\begin{frame}[label={sec:orgc494eea},fragile,plain]{Variables locales}
 \lstset{language=C,label= ,caption= ,captionpos=b,numbers=none}
\begin{lstlisting}
#include <stdio.h>

void foo(void);

int main()
{
  int x = 1; // variable local en main
  printf("main (1):\t x es %d.\n", x);
  foo();
  printf("main (2):\t x es %d.\n", x);
  return 0;
}

void foo(void)
{
  int x = 2; // variable local en foo
  printf("foo:\t x es %d.\n", x);
}
\end{lstlisting}
\end{frame}

\section{Funciones que llaman a otras funciones}
\label{sec:org1ef430d}

\begin{frame}[label={sec:org3ca1f6d},fragile,plain]{Ejemplo}
 \lstset{language=C,label= ,caption= ,captionpos=b,numbers=none}
\begin{lstlisting}
#include <stdio.h>
#define PI 3.141592

float eleva3(float x);
float volEsfera(float r);

int main(){
  float radio, vol;
  scanf("%f", &radio);
  vol = volEsfera(radio);
  printf("El volumen es %f", vol);
  return 0;
}
float volEsfera(float r){ //Usa eleva3
  return 4.0/3.0 * PI * eleva3(r);
}
float eleva3(float x){
  return x * x * x;
}
\end{lstlisting}
\end{frame}
\begin{frame}[label={sec:orgc1f236f},fragile,plain]{Funciones recursivas}
 \lstset{language=C,label= ,caption= ,captionpos=b,numbers=none}
\begin{lstlisting}
#include <stdio.h>

int fact(int n);

int main(){
  int x;
  printf("Indica un número:\n");
  scanf("%d", &x);
  printf("El factorial de %d es %d\n", x, fact(x));
  return 0;
}
int fact(int n){
  int res;
  if (n > 1) // Incluye llamada a si misma
    res = n * fact(n - 1);
  else
    res = 1;
  return res;
}
\end{lstlisting}
\end{frame}

\section{Funciones en ficheros}
\label{sec:orgb015608}
\begin{frame}[label={sec:org2b6659c},fragile]{Motivación y uso}
 \begin{block}{Motivación}
Para poder reutilizar las funciones definidas es conveniente alojarlas
en un fichero (o colección de ficheros) que puedan ser incluidos en
otros proyectos.
\end{block}

\begin{block}{Uso}
\begin{itemize}
\item Debe existir un (o varios) fichero(s) \texttt{.h} (cabecera) y un fichero \texttt{.c} (código fuente, implementación de las funciones).
\item Se debe usar \texttt{\#include "nombre\_lib.h"} al comienzo del programa.
\item Hay que compilar conjuntamente (\emph{en un proyecto}).
\end{itemize}
\end{block}
\end{frame}

\begin{frame}[label={sec:org9dc3dd9},fragile]{Ejemplo (1)}
 \begin{block}{Fichero \texttt{myLib.h} (cabecera)}
\lstset{language=C,label= ,caption= ,captionpos=b,numbers=none}
\begin{lstlisting}
#define PI 3.141592

float eleva3(float x);
float volEsfera(float r);
\end{lstlisting}
\end{block}

\begin{block}{Fichero \texttt{myLib.c} (código fuente)}
\lstset{language=C,label= ,caption= ,captionpos=b,numbers=none}
\begin{lstlisting}
#include "myLib.h"

float volEsfera(float r){
  return 4.0/3.0 * PI * eleva3(r);
}
float eleva3(float x){
  return x * x * x;
}
\end{lstlisting}
\end{block}
\end{frame}

\begin{frame}[label={sec:orgaea0b84},fragile]{Ejemplo (2)}
 \begin{block}{Programa principal}
\lstset{language=C,label= ,caption= ,captionpos=b,numbers=none}
\begin{lstlisting}
#include <stdio.h>
// Directiva para incluir la librería local
#include "myLib.h" 

int main()
{
  float radio, vol;
  scanf("%f", &radio);
  vol = volEsfera(radio);
  printf("El volumen es %f", vol);
  return 0;
}
\end{lstlisting}
\end{block}
\end{frame}
\end{document}