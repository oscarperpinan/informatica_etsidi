% Created 2023-02-17 vie 11:09
% Intended LaTeX compiler: pdflatex
\documentclass[usenames,svgnames,dvipsnames, aspectratio=169]{beamer}
\usepackage[utf8]{inputenc}
\usepackage[T1]{fontenc}
\usepackage{graphicx}
\usepackage{longtable}
\usepackage{wrapfig}
\usepackage{rotating}
\usepackage[normalem]{ulem}
\usepackage{amsmath}
\usepackage{amssymb}
\usepackage{capt-of}
\usepackage{hyperref}
\usepackage{color}
\usepackage{listings}
\usepackage[spanish]{babel}
\usecolortheme{rose}
\setbeamercolor{alerted text}{fg=DarkBlue}
\setbeamerfont{alerted text}{series=\bfseries}
\setbeamerfont{block title}{series=\bfseries}
\setbeamercolor{block title}{bg=structure.fg!20!bg!50!bg}
\setbeamercolor{block body}{use=block title,bg=block title.bg}
\setbeamertemplate{navigation symbols}{\insertsectionnavigationsymbol}
\AtBeginSection[]{\begin{frame}[plain]\tableofcontents[currentsection,sectionstyle=show/shaded]\end{frame}}
\AtBeginSubsection[]{\begin{frame}[plain]\tableofcontents[currentsubsection,sectionstyle=show/shaded,subsectionstyle=show/shaded]\end{frame}}
\lstset{keywordstyle=\color{blue}, commentstyle=\color{gray!90}, basicstyle=\ttfamily\small, columns=fullflexible, breaklines=true,linewidth=\textwidth, backgroundcolor=\color{gray!23}, basewidth={0.5em,0.4em}, literate={¡}{{\textexclamdown}}1 {á}{{\'a}}1 {ñ}{{\~n}}1 {é}{{\'e}}1 {ó}{{\'o}}1 {í}{{\'i}}1 {ú}{{\'u}}1 {º}{{\textordmasculine}}1, showstringspaces=false}
\usepackage{mathpazo}
\usepackage{siunitx}
\hypersetup{colorlinks=true, linkcolor=Blue, urlcolor=Blue}
\usepackage{fancyvrb}
\DefineVerbatimEnvironment{verbatim}{Verbatim}{fontsize=\tiny, formatcom = {\color{black!70}}}
\setbeamertemplate{footline}[frame number]
\usetheme{Boadilla}
\usefonttheme{serif}
\author{Oscar Perpiñán Lamigueiro}
\date{}
\title{Tema 3: Sentencias de Control}
\hypersetup{
 pdfauthor={Oscar Perpiñán Lamigueiro},
 pdftitle={Tema 3: Sentencias de Control},
 pdfkeywords={},
 pdfsubject={},
 pdfcreator={Emacs 28.2 (Org mode 9.6)}, 
 pdflang={Spanish}}
\begin{document}

\maketitle

\section{Introducción}
\label{sec:org068ba66}
\begin{frame}[label={sec:org5f04e3b},fragile]{Introducción}
 \begin{itemize}
\item Sin sentencias de control los programas se ejecutan de manera secuencial
\end{itemize}
\begin{block}{Sentencias de Control}
\begin{description}
\item[{Sentencias Condicionales}] ejecutan unas secuencias u otras según el cumplimiento de unas condiciones.
\begin{itemize}
\item \texttt{if}, \texttt{if - else}
\item \texttt{switch - case}
\end{itemize}
\item[{Sentencias Repetitivas}] repiten un conjunto de sentencias en unas determinadas condiciones.
\begin{itemize}
\item \texttt{for}
\item \texttt{while}, \texttt{do - while}
\end{itemize}
\end{description}
\end{block}
\end{frame}
\section{Sentencias Condicionales}
\label{sec:org77dc3df}
\subsection{\texttt{if - else}}
\label{sec:org80e75e6}
\begin{frame}[label={sec:orgec1ee0c},fragile]{\texttt{if}}
 \begin{center}
\includegraphics[height=0.9\textheight]{figs/if.pdf}
\end{center}
\end{frame}

\begin{frame}[label={sec:orge58f450},fragile]{\texttt{if}}
 \begin{itemize}
\item Si se cumple \texttt{condicion} (su resultado es \alert{diferente de 0}) se ejecuta la \texttt{sentencia\_A}.
\item \alert{Siempre} se ejecuta la \texttt{sentencia\_B}.
\end{itemize}
\begin{lstlisting}[language=C,label= ,caption= ,captionpos=b,numbers=none]
if (condicion)
    sentencia_A;
sentencia_B;
\end{lstlisting}
\begin{itemize}
\item Para ejecutar un conjunto de sentencias hay que agruparlas \alert{entre llaves}.
\end{itemize}
\begin{lstlisting}[language=C,label= ,caption= ,captionpos=b,numbers=none]
if (condicion)
  {
    sentencia_A1;
    sentencia_A2;
    ...
  }
sentencia_B1;
sentencia_B2;
...
\end{lstlisting}
\end{frame}

\begin{frame}[label={sec:orgfb44ecf},fragile]{Ejemplo \texttt{if}}
 \begin{lstlisting}[language=C,label= ,caption= ,captionpos=b,numbers=none]
# include <stdio.h>
int main ()
{
  int n;
  printf("Escribe un número entero\n");
  scanf("%d", &n);
  if (n % 2 == 0) // Condición
    {// Uso de llaves
      printf("Se cumple la condición: ");
      printf("El número %d es par.\n", n);      
    } // Fin de if
  printf("Gracias por participar.\n");
  return 0;
}
\end{lstlisting}
\end{frame}

\begin{frame}[label={sec:orga2b0018},fragile]{Ejemplo \texttt{if}}
 \begin{lstlisting}[language=C,label= ,caption= ,captionpos=b,numbers=none]
# include <stdio.h>
int main ()
{
  int n;
  printf("Escribe un número entero\n");
  scanf("%d", &n);
  if (n % 2 == 0) // Sin llaves para sentencias simples
    printf("El número %d es par.\n", n);
  // Fin de if
  printf("Gracias por participar.\n");
  return 0;
}
\end{lstlisting}
\end{frame}


\begin{frame}[label={sec:org5b30bb7},fragile]{Ejemplo \texttt{if}}
 \begin{lstlisting}[language=C,label= ,caption= ,captionpos=b,numbers=none]
# include <stdio.h>
int main ()
{
  int n;
  printf("Escribe un número entero\n");
  scanf("%d", &n);
  if (n % 2) //Esta condición no es un booleano
    //La sentencia se ejecuta cuando la condición
    //*no* es igual a cero (n % 2 != 0)
    printf("El número %d es impar.\n", n);
  printf("Gracias por participar.\n");
  return 0;
}
\end{lstlisting}
\end{frame}



\begin{frame}[label={sec:org719734a},fragile]{\texttt{if - else}}
 \begin{center}
\includegraphics[height=0.9\textheight]{figs/ifelse.pdf}
\end{center}
\end{frame}


\begin{frame}[label={sec:orgfa018e2},fragile]{\texttt{if - else}}
 \begin{description}
\item[{\texttt{if}}] Si se cumple \texttt{condicion} (su resultado es \alert{diferente de 0}) se ejecuta la \texttt{sentencia\_A}.
\item[{\texttt{else}}] En caso contrario se ejecuta la \texttt{sentencia\_B}.
\item[{Siempre}] se ejecuta la \texttt{sentencia\_C}.
\end{description}
\begin{lstlisting}[language=C,label= ,caption= ,captionpos=b,numbers=none]
if (condicion)
    sentencia_A;
else
    sentencia_B;
sentencia_C;
\end{lstlisting}
\end{frame}
\begin{frame}[label={sec:orgf3248c5},fragile]{\texttt{if - else}}
 \begin{itemize}
\item Para ejecutar un conjunto de sentencias hay que agruparlas \alert{entre llaves}.
\end{itemize}
\begin{lstlisting}[language=C,label= ,caption= ,captionpos=b,numbers=none]
if (condicion)
  {
    sentencia_A1;
    sentencia_A2;
    ...
  }
else
  {
    sentencia_B1;
    sentencia_B2;
    ...    
  }
sentencia_C1;
sentencia_C2;
...
\end{lstlisting}
\end{frame}


\begin{frame}[label={sec:org7581f8a},fragile,plain]{Ejemplo \texttt{if-else}}
 \begin{lstlisting}[language=C,label= ,caption= ,captionpos=b,numbers=none]
# include <stdio.h>
int main (){
  int n;
  printf("Escribe un número entero\n");
  scanf("%d", &n);
  if (n % 2 == 0) // condicion
    { // Inicio de if 
      printf("Se cumple la condición: ");
      printf("El número %d es par.\n", n);      
    }
  else
    { // Inicio de else
      printf("No se cumple la condición: ");
      printf("El número %d es impar.\n", n);      
    } // Fin de if - else
  printf("Gracias por participar.\n");
  return 0;
}
\end{lstlisting}
\end{frame}

\begin{frame}[label={sec:orgb49c50c},fragile]{Ejemplo \texttt{if-else}}
 \begin{lstlisting}[language=C,label= ,caption= ,captionpos=b,numbers=none]
# include <stdio.h>
int main ()
{
  int n;
  printf("Escribe un número entero\n");
  scanf("%d", &n);
  if (n % 2 == 0)
    printf("El número %d es par.\n", n);
  else
    printf("El número %d es impar.\n", n);
  printf("Gracias por participar.\n");
  return 0;
}
\end{lstlisting}
\end{frame}


\begin{frame}[label={sec:org3948b31},fragile]{\texttt{if - else - if}}
 \begin{center}
\includegraphics[height=0.9\textheight]{figs/ifelseif.pdf}
\end{center}
\end{frame}
\begin{frame}[label={sec:org4c9ea69},fragile]{\texttt{if - else - if}}
 \begin{description}
\item[{\texttt{if}}] Si se cumple \texttt{condicion\_1} se ejecuta la \texttt{sentencia\_A}.
\item[{\texttt{else}}] En caso contrario \ldots{}
\begin{description}
\item[{\texttt{if}}] si se cumple \texttt{condicion\_2} se ejecuta la \texttt{sentencia\_B}.
\item[{\texttt{else}}] En caso contrario se ejecuta \texttt{sentencia\_C}.
\end{description}
\item[{Siempre}] se ejecuta la \texttt{sentencia\_D}.
\end{description}
\begin{lstlisting}[language=C,label= ,caption= ,captionpos=b,numbers=none]
if (condicion_1)
    sentencia_A;
else
  if (condicion_2)
    sentencia_B;
  else
    sentencia_C;
sentencia_D;
\end{lstlisting}
\end{frame}


\begin{frame}[label={sec:org963a706},fragile]{\texttt{if - else - if}}
 \begin{description}
\item[{\texttt{if}}] Si se cumple \texttt{condicion\_1} se ejecuta la \texttt{sentencia\_A}.
\item[{\texttt{else if}}] En caso contrario, si se cumple \texttt{condicion\_2} se ejecuta la \texttt{sentencia\_B}.
\begin{description}
\item[{\texttt{else}}] En caso contrario se ejecuta \texttt{sentencia\_C}.
\end{description}
\item[{Siempre}] se ejecuta la \texttt{sentencia\_D}.
\end{description}
\begin{lstlisting}[language=C,label= ,caption= ,captionpos=b,numbers=none]
if (condicion_1)
    sentencia_A;
else if (condicion_2)
    sentencia_B;
else
   sentencia_C;
sentencia_D;
\end{lstlisting}
\end{frame}

\begin{frame}[label={sec:org4a88002},fragile,plain]{Ejemplo \texttt{if - else - if}}
 \begin{lstlisting}[language=C,label= ,caption= ,captionpos=b,numbers=none]
#include <stdio.h>

int main(){
  int x;
  printf("Escribe un número: ");
  scanf("%i", &x);

  if (x < 0) 
    // se cumple la primera condición
    printf("El número es negativo.\n");
  else if (x == 0)
    // se cumple la segunda
    printf("El número es 0.\n");
  else
    // no se cumple ninguna
    printf("El número es positivo.\n");
  return 0;
}
\end{lstlisting}
\end{frame}

\subsection{\texttt{switch - case}}
\label{sec:org22e1169}

\begin{frame}[label={sec:org34e3c5b},fragile]{\texttt{switch-case}}
 \begin{itemize}
\item Permite tomar una decisión múltiple dependiendo del valor \alert{entero} de una expresión.
\end{itemize}

\begin{center}
\includegraphics[height=0.8\textheight]{figs/switch.pdf}
\end{center}
\end{frame}


\begin{frame}[label={sec:orgc7efee4},fragile]{\texttt{switch-case}}
 \begin{itemize}
\item Permite tomar una decisión múltiple dependiendo del valor \alert{entero} de una expresión.
\end{itemize}
\begin{lstlisting}[language=C,label= ,caption= ,captionpos=b,numbers=none]
switch (expr)
  {
  case val1:
    sentencia_1;
    break;
  case val2:
    sentencia_2;
    break;
  case val3:
    sentencia_3;
    break;
  ...
  default:
    sentencia_n;
    break;
  }
\end{lstlisting}
\end{frame}
\begin{frame}[label={sec:orgb4ba0fe},fragile,plain]{Ejemplo de \texttt{switch - case}}
 \begin{lstlisting}[language=C,label= ,caption= ,captionpos=b,numbers=none]
#include <stdio.h>
int main () {
  float v1, v2;
  char op;
  scanf("%f %c %f", &v1, &op, &v2);
  switch(op)
    {
    case '+':
      printf("%.2f\n", v1 + v2);
      break;
    case '-':
      printf("%.2f\n", v1 - v2);
      break;
    default:
      printf("No se hacer esa operación.\n");
      break;
    }
  return 0; 
}
\end{lstlisting}
\end{frame}

\begin{frame}[label={sec:orgab357c7},fragile]{Atención al uso de \texttt{break}}
 \begin{lstlisting}[language=C,label= ,caption= ,captionpos=b,numbers=none]
#include <stdio.h>
int main ()
{
  float v1, v2;
  char op;
  scanf("%f %c %f", &v1, &op, &v2);
  switch(op)
    {
    case '+':
      printf("%.2f\n", v1 + v2);
    case '-':
      printf("%.2f\n", v1 - v2);
    default:
      printf("No se hacer esa operación.\n");
      break;
    }
  return 0;
}
\end{lstlisting}
\end{frame}
\begin{frame}[label={sec:org8a926ef},fragile,plain]{Uso de llaves con \texttt{switch}}
 \begin{lstlisting}[language=C,label= ,caption= ,captionpos=b,numbers=none,basicstyle=\ttfamily\footnotesize]
#include <stdio.h>
int main (){
  float v1, v2;
  char op;
  scanf("%f %c %f", &v1, &op, &v2);
  switch(op)
    {
    case '+':
      printf("Operación Suma:\n");
      printf("%.2f\n", v1 + v2);
      break;
    case '-':
      printf("Operación Resta:\n");
      printf("%.2f\n", v1 - v2);
      break;
    default:
      printf("No se hacer esa operación.\n");
      break;
    }
  return 0;
}
\end{lstlisting}
\end{frame}
\section{Sentencias Repetitivas (Bucles)}
\label{sec:orgdec94f7}
\subsection{\texttt{for}}
\label{sec:org114826c}

\begin{frame}[label={sec:org4c5f319},fragile]{\texttt{for}: Flujo}
 Ejecuta una sentencia (simple o compuesta) un número determinado de veces hasta que el resultado de una expresión sea falso.

\begin{center}
\includegraphics[height=0.8\textheight]{figs/for.pdf}
\end{center}
\end{frame}

\begin{frame}[label={sec:orgdce0706},fragile]{\texttt{for} : Código}
 Ejecuta una sentencia (simple o compuesta) un número determinado de veces hasta que el resultado de una expresión sea falso.

\begin{lstlisting}[language=C,label= ,caption= ,captionpos=b,numbers=none]
for (expr_inicio; expr_final; expr_avance)
  {
      sentencia_1;
      ...
  }
\end{lstlisting}

\begin{description}
\item[{\texttt{expr\_inicio}}] Expresión de inicialización (se ejecuta una sola vez). Sirve para iniciar las variables de control del bucle.
\item[{\texttt{expr\_final}}] Expresión numérica, relacional o lógica. Si es falsa se acaba el bucle (si se omite, se considera siempre verdadera y, por tanto, bucle infinito).
\item[{\texttt{expr\_avance}}] Expresión (o expresiones separadas por comas) de progresión del bucle.
\end{description}
\end{frame}

\begin{frame}[label={sec:org2f7ac2f},fragile]{Ejemplo: suma de enteros}
 \begin{lstlisting}[language=C,label= ,caption= ,captionpos=b,numbers=none]
#include <stdio.h>

int main()
{
  // Todas las variables deben estar definidas
  // Asigno valor inicial 0 a suma
  int i, suma = 0, n = 10;

  for (i = 1; i <= n; i++)
    {
      suma += i;
    }
  printf("La suma de los %d primeros enteros es %d",
	 n, suma);
  return 0;
}
\end{lstlisting}
\end{frame}

\begin{frame}[label={sec:orgd1ae7b5},fragile]{Ejemplo: factorial}
 \begin{lstlisting}[language=C,label= ,caption= ,captionpos=b,numbers=none]
#include <stdio.h>

int main()
{
  int i, n = 20;
  // El factorial alcanza valores grandes y siempre es positivo.
  unsigned long int fact;
  // La expresión de inicio puede ser múltiple,
  // separando por comas.
  // Aquí asigno valor inicial a fact
  for (i = 1, fact = 1; i <= n; i++)
      fact *= i;

  printf("El factorial de %d es %lu", 
	 n, fact);
  return 0;
}
\end{lstlisting}
\end{frame}


\begin{frame}[label={sec:orgbc17d9e},fragile]{Ejemplo: alfabeto}
 \begin{lstlisting}[language=C,label= ,caption= ,captionpos=b,numbers=none]
#include <stdio.h>

int main()
{
  char i;
  // Se pueden usar char en las expresiones 
  for (i = 'a'; i <= 'z'; i++)
      printf("%c", i);
  return 0;
}
\end{lstlisting}
\end{frame}


\begin{frame}[label={sec:orgeee50d8},fragile]{Ejemplo: bucles anidados}
 \begin{lstlisting}[language=C,label= ,caption= ,captionpos=b,numbers=none]
#include <stdio.h>

int main()
{
  int i, j;
  for (i = 1; i <= 10; i++)
    {//Atención al uso de las llaves
      printf("Tabla del %d\n", i);

      for (j = 1; j <= 10; j++)
	printf("%d x %d = %d\n",
	       i, j, i * j);
    }
  return 0;
}
\end{lstlisting}
\end{frame}

\subsection{\texttt{while}}
\label{sec:orgec69567}

\begin{frame}[label={sec:org52cf7fd},fragile]{\texttt{while}: Flujo}
 Ejecuta una sentencia (simple o compuesta) \alert{cero o más veces} dependiendo del resultado booleano de una expresión.

\begin{center}
\includegraphics[height=0.8\textheight]{figs/while.pdf}
\end{center}
\end{frame}

\begin{frame}[label={sec:org4f23728},fragile]{\texttt{while}: Código}
 Ejecuta una sentencia (simple o compuesta) \alert{cero o más veces} dependiendo del resultado booleano de una expresión.
\begin{lstlisting}[language=C,label= ,caption= ,captionpos=b,numbers=none]
while (expresion)
  {
      sentencia_1;
      sentencia_2;
      ...
  }

\end{lstlisting}
\end{frame}

\begin{frame}[label={sec:org0df5926},fragile]{\texttt{while}: Atención}
 \begin{block}{}
\begin{itemize}
\item Si la primera vez que se evalúa la condición es falsa, el bloque de sentencias \alert{no se ejecuta nunca}.

\item Si la expresión es siempre verdadera \alert{el bucle es infinito} (dentro de la sentencia debe haber una instrucción que modifique su estado).
\end{itemize}
\end{block}
\end{frame}

\begin{frame}[label={sec:orgfd92a92},fragile]{Ejemplo}
 \begin{lstlisting}[language=C,label= ,caption= ,captionpos=b,numbers=none]
#include <stdio.h>

int main()
{
  int i;

  i = 5;
  while (i > 0)
    {
      printf("%d...", i);
      --i;
    }
  printf("Despegue!");
  return 0;
}
\end{lstlisting}
\end{frame}

\subsection{\texttt{do-while}}
\label{sec:org068b813}

\begin{frame}[label={sec:orgfaa87a7},fragile]{\texttt{do-while}: Flujo}
 Ejecuta una sentencia (simple o compuesta) \alert{una o más veces} dependiendo del resultado de una expresión.

\begin{center}
\includegraphics[height=0.8\textheight]{figs/do_while.pdf}
\end{center}
\end{frame}

\begin{frame}[label={sec:orgb86f537},fragile]{\texttt{do-while}: Código}
 Ejecuta una sentencia (simple o compuesta) \alert{una o más veces} dependiendo del resultado de una expresión.
\begin{lstlisting}[language=C,label= ,caption= ,captionpos=b,numbers=none]
do
{
  sentencia_1;
  sentencia_2;
  ...
}
while (expresion);
\end{lstlisting}
\end{frame}


\begin{frame}[label={sec:org198fb06},fragile]{\texttt{do-while}: Atención}
 \begin{block}{}
\begin{itemize}
\item Si la primera vez que se evalúa la expresión es falsa, la sentencia se habrá ejecutado al menos una vez.

\item Si la expresión es siempre verdadera \alert{el bucle es infinito} (dentro de la sentencia debe haber una instrucción que modifique su estado)
\end{itemize}
\end{block}
\end{frame}

\begin{frame}[label={sec:org236f529},fragile]{Ejemplo: número entero al revés}
 \begin{lstlisting}[language=C,label= ,caption= ,captionpos=b,numbers=none]
#include <stdio.h>

int main()
{
  int num = 123456, cifra;

  do
    {
      cifra = num % 10;
      printf("%d", cifra);
      num = num / 10;
    }
  while (num > 0);
  printf("\n");
  return 0;
}
\end{lstlisting}
\end{frame}


\subsection{Equivalencia}
\label{sec:orgc987b55}

\begin{frame}[label={sec:org8ed4561},fragile,plain]{Equivalencia}
 \begin{lstlisting}[language=C,label= ,caption= ,captionpos=b,numbers=none,basicstyle=\ttfamily\footnotesize]
#include <stdio.h>
int main() {
  int i,n = 10;
  // Bucle for
  for (i = 0; i <= n; i++)
      printf("%d ",i);
  // Bucle while
  i=0;
  while( i<= n) {
      printf("%d ",i);
      i++;
    }
  // Bucle do-while
  i=0;
  do {
      printf("%d ",i);
      i++;
    }
  while(i <= n);
  return 0;
}
\end{lstlisting}
\end{frame}
\begin{frame}[label={sec:orgfe86a0d},fragile]{¿Qué bucle elegir?}
 \begin{itemize}
\item Si se conoce el número de veces que debe ejecutarse la tarea es recomendable usar \texttt{for}.
\item Si el número de veces es desconocido a priori:
\begin{itemize}
\item Si debe realizarse al menos una vez se debe usar \texttt{do-while}.
\item Si no es imprescindible que se ejecute alguna vez, se puede usar \texttt{while}.
\end{itemize}
\end{itemize}
\end{frame}
\section{Rupturas}
\label{sec:org61cfebb}
\begin{frame}[label={sec:orgd4afbfc},fragile]{\texttt{break} y \texttt{continue}}
 \begin{block}{\texttt{break}}
\begin{itemize}
\item Finaliza la ejecución de un bucle (si el bucle está anidado sólo finaliza él, pero no los bucles más externos).
\end{itemize}
\end{block}

\begin{block}{\texttt{continue}}
\begin{itemize}
\item Ejecuta la siguiente iteración del bucle.
\item En un bucle \texttt{while} o \texttt{do-while} vuelve a \texttt{expresion}.
\item En un bucle \texttt{for} ejecuta \texttt{expr\_avance} y a continuación comprueba \texttt{expr\_final}
\end{itemize}
\end{block}
\end{frame}
\end{document}